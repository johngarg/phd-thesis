% path to figures directory
\graphicspath{{img/chapter_6/}}

\chapter{Conclusions and Outlook}
\label{chapter:conclusions}

The strongest indication of physics beyond the Standard Model is the clear and
overwhelming evidence for neutrino oscillations and the non-zero neutrino masses
these observations imply. Unfortunately there are many viable scenarios for
explaining the origin of mass and mixing in the neutrino sector, and it is not
clear which models should receive more attention. The simplest tend to be far
beyond the current or near-future reach of experiments, at least in their most
motivated regions of parameter space. There are many non-minimal but testable
models one can construct that explain the smallness of the neutrino masses in a
satisfying way, although studying these models systematically has proven to be
difficult. In this thesis we have mapped out the space of one very motivated
class of such models: minimal models of Majorana neutrino mass. In doing so we
hope to have provided a platform for various systematic studies, perhaps both
phenomenological and experimental. We have also begun to explore the potential
role these models might play in underlying the exciting but still illusive
$B$-meson anomalies, whose connection to neutrino physics was an unexpected but
welcome development that occurred in the course of this doctoral work.

In Chapter~\ref{chapter:mv-models} we presented the algorithm and computational
machinery we used to generate the roughly eleven thousand models that constitute
our model database. Almost all of the models have the neutrinos picking up mass
at loop level, and therefore the database is essentially one of radiative models
of Majorana neutrino mass. Our analysis was built on lepton-number-violating
effective operators, many of which appear for the first time in our study. The
models are represented in a computational format designed to facilitate future
automated phenomenological analysis, and we have made both the code we used to
implement the algorithm as well as the database publicly available. We conducted
a preliminary study on the basis of the database, finding a number of
interesting models containing relatively few free parameters while predicting
very low-scale new physics. We also investigated the structure of the
neutrino-mass matrices of some novel models, including one derived from an
operator containing derivatives, and a four-multiplet model incorporating a
specific explanation of the flavour anomalies. The model database is a fantastic
tool for identifying different motifs and patterns across the space of Majorana
neutrino mass models. One important finding of our study was the preponderance
of scalar leptoquarks in the completions.

The recent experimental hints of lepton flavour non-universality in neutral- and
charged-current $B$ decays have ignited interest in phenomenological models of
scalar leptoquarks, since these have been identified as strong candidates that
could underlie the discrepant measurements. In Chapter~\ref{chapter:the-one-lq}
we detailed a comprehensive study of a particularly simple such scenario,
originally proposed in Ref.~\cite{Bauer:2015knc}. The so-called \textit{one
  leptoquark} model presents the $S_{1}$ leptoquark as an economic explanation
of both the $b \to s$ and $b \to c$ anomalies through related sets of Yukawa
couplings to left-handed SM fermions. The results of our detailed
phenomenological analysis of the model showed that its viability was compromised
by a full consideration of its constraints. The ability of the leptoquark to
explain the measured values of $R_{D}$ and $R_{D^{*}}$ remained, although in a
previously unconsidered region of parameter space entailing Yukawa couplings to
both left- and right-handed SM fermions. A combined explanation of both sets of
anomalies was found to be possible, along with a mechanism to explain the
discrepant value of the anomalous magnetic moment of the muon. In this combined
scenario, we found that the $b \to s$ and $b \to c$ anomalies could be explained
to within $2\sigma$ while also accommodating the measurement of $(g-2)_{\mu}$,
an impressive improvement on the SM for such a simple model.

The analysis we conducted on our model database suggests that there may be a
connection between radiative models of neutrino mass and the flavour anomalies.
This is a phenomenologically rich research direction that we have only begun to
explore, and our work on this topic is described in
Chapter~\ref{chapter:neutrino-mass-and-flavour-anomalies}. There, we studied two
models of radiative neutrino mass involving scalar leptoquarks. The first
contained the $S_{1}$ leptoquark; we explored the extent to which the pattern of
neutrino masses and mixings observed could be reproduced in this model, while at
the same time accommodating the flavour anomalies. We found that the structure
of the mass matrix was such that the problematic coupling of $S_{1}$ to the
electron could be arranged to be small at the cost of fixing the ratio of the
tau--top $S_{1}$ coupling to the muon--top $S_{1}$ coupling. This implied very
clear predictions of tau--muon LFV observables like $\tau \to \mu \gamma$ and
$\tau \to \mu\mu\mu$ if the muon couplings were at all sizeable. Large couplings
to the muon are necessary in this model to explain the $b \to s$ anomalies and
so these could not be explained along with the neutrino masses without
unacceptably large rates for $\tau \to \mu$ observables.

Given this simple model's limited success, we also considered a next-to-minimal
scenario in which the mechanisms explaining the charged- and neutral-current
anomalies were somewhat divorced. The model we studied is a completion of a
dimension-seven operator containing the $S_{3}$ leptoquark, known to be provide
a good explanation of the neutral-current anomalies, and the vector-like quark
$\chi$. We add $S_{1}$ to the model in order to incorporate the successes of the
\textit{one leptoquark} in explaining the anomalies in $R_{D}$, $R_{D^{*}}$ and
$(g-2)_{\mu}$. In this model, the coupling to the electron could not be avoided
on account of the structure of the neutrino-mass matrix. This implied a
parameter space constrained most strongly by muon--electron LFV observables, the
most important of which we found to be muon--electron conversion in nuclei.
Interestingly, our analysis showed an intimate connection between the
muon--electron conversion rate, the size of the
$C_{9}^{bs\mu\mu} = - C_{10}^{bs\mu\mu}$ Wilson coefficients, and even the value
of the Majorana phase $\alpha_{2}$. Excitingly, the model predicts a
muon--electron conversion rate that will be observed or excluded by the next
generation of experiments, along with a $D^{0} \to \mu\mu$ rate an order of
magnitude larger than the SM prediction. The model can successfully explain all
of the discrepant measurements along with neutrino mass, showing that a combined
explanation is possible.

We expect our model database to be a useful resource for exploring the
connection radiative models might have to the flavour anomalies. To facilitate
future work in this direction, we finished
Chapter~\ref{chapter:neutrino-mass-and-flavour-anomalies} with a selection of
models from the database that contain particle content sufficient to explain the
anomalies. These models predict a tighter connection between the anomalies and
neutrino physics than seen in the next-to-minimal scenario studied above, since
each field is necessary for the violation of lepton number.

We eagerly await the results of future measurements, particularly of the clean
ratios $R_{K}$ and $R_{K^{*}}$, to see what they may imply about these models
and the connection to radiative neutrino mass going forward. Should they show
that the anomalies were really some statistical or experimental effect, it would
not be the first time this has happened. Indeed, in Chapter~\ref{chapter:quirk}
a remarkably simple explanation of the \SI{750}{\GeV} diphoton excess was
introduced, and this anomaly has since vanished. The model introduced a scalar
field charged under a new confining $\mathrm{SU}(N)$, the diphoton decay of
whose bound state $\digamma$ explained the excess seen by ATLAS and CMS. An
interesting feature of the model is that pair production of the scalar and the
subsequent formation of the bound state dominates over the direct $\digamma$
production, since there are no light $\mathrm{SU}(N)$-charged states that can be
produced from the vacuum to break the $\mathrm{SU}(N)$ string.

This is perhaps a good example of scientific creativity and useful model
building, albeit in response to spurious new physics. Should some or all of the
flavour anomalies also go like $\digamma$, we hope that the same can be said of
our research connecting radiative neutrino masses and the flavour anomalies.
Indeed, in this case the main results of this doctoral work will essentially
remain unchanged: models explaining deviations in dimension-six four-fermion
operators will always have some possible connection to models of radiative
neutrino mass. Now that, subject to certain minimality assumptions, these have
been catalogued in our model database, it remains for experimentalists to
continue to probe the coefficients at dimension six, and for phenomenologists to
continue to constrain and draw out predictions from the models.
