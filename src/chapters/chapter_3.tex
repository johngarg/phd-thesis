% path to figures directory
\graphicspath{{img/chapter_3/}}


\chapter{The $S_{1}$ leptoquark as an explanation of the flavour anomalies}
\label{chapter:the-one-lq}

\begin{flushleft}
  \textit{This chapter is based on the publication `Reconsidering the One
    Leptoquark scenario: flavour anomalies and neutrino mass,' written in
    collaboration with Yi Cai, Michael A. Schmidt, and Raymond R.
    Volkas~\cite{Cai:2017wry}. We study the potential of the $S_{1}$ leptoquark
    to explain the flavour anomalies and the anomalous magnetic moment of the
    muon in a new region of parameter space.}
\end{flushleft}

\section{Introduction}

A common origin for $R_{D^{(*)}}$ and the anomalous $b\rightarrow s$ data is
suggested naturally if the former is explained by the effects of the operator
$(c^{\dagger} \bar{\sigma}_\mu b)(\tau^{\dagger} \bar{\sigma}^\mu \nu)$, related
in its general structure by $\mathrm{SU}(2)_L$ invariance to the aforementioned
four-fermion effective operator accounting for the $b \rightarrow s$ anomalies.
A number of models exploring this idea have been suggested in the
literature~\cite{Alonso:2015sja, Bauer:2015knc, Becirevic:2016oho,
  Becirevic:2016yqi, Boucenna:2016wpr, Boucenna:2016qad, Calibbi:2015kma,
  Crivellin:2017zlb, Deppisch:2016qqd, Deshpand:2016cpw, Fajfer:2015ycq,
  Feruglio:2016gvd, Feruglio:2017rjo, Megias:2017ove, Popov:2016fzr} (along with
many others addressing one or the other anomaly,
\textit{e.g.}~\cite{Becirevic:2015asa, Becirevic:2017jtw, Buras:2013qja,
  Freytsis:2015qca, Gauld:2013qba, Glashow:2014iga, Gripaios:2014tna,
  Hiller:2014ula, Hiller:2014yaa, Mahmoudi:2014mja, Megias:2016bde, Pas:2015hca,
  Sahoo:2015fla, Sahoo:2015qha, Sakaki:2013bfa, Sierra:2015fma,
  Varzielas:2015iva, deBoer:2015boa}) and among these minimal explanations the
Bauer--Neubert (BN) model~\cite{Bauer:2015knc} is one of notable simplicity and
explanatory power: a \TeV-scale scalar leptoquark protagonist mediating
$B \rightarrow D^{(*)} \tau \nu$ at tree-level and the $b \rightarrow s$ decays
through one-loop box diagrams. The leptoquark transforms under the SM gauge
group like a right-handed down-type quark and its pattern of couplings to SM
fermions can also reconcile the measured and predicted values of the anomalous
magnetic moment of the muon.

Our aim in this chapter is to study the scalar-leptoquark model in the context
of some previously unconsidered constraints and comment more definietely on its
viability as an explanation of both the charged- and neutral-current anomalies.
