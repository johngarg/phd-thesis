% path to figures directory
\graphicspath{{img/chapter_4/}}

\chapter{The two-photon decay of a scalar-quirk bound state}
\label{chapter:quirk}

\begin{flushleft}
  \textit{\lipsum[1]}
\end{flushleft}

\section{Introduction}

An excess of events containing two photons with invariant mass near
\SI{750}{\GeV} has been observed in 13 TeV proton--proton collisions by the
ATLAS and CMS collaborations~\cite{ATLAS-CONF-2015-081, CMS:2015dxe}.
The cross section $\sigma(pp \rightarrow \gamma \gamma)$ is estimated to be
\begin{align}
  \begin{split}
    \sigma (p p \rightarrow \gamma \gamma) &=
    \begin{cases}
      (10 \pm 3)~\fb & \text{ATLAS} \\
      (6 \pm 3)~\fb & \text{CMS}
    \end{cases}
  \end{split}
\end{align}
and there is no evidence of any accompanying excess in the dilepton
channel~\cite{ATLAS-CONF-2015-070}. If we interpret this excess as the two
photon decay of a single new particle of mass $m$ then ATLAS data provide a hint
of a large width: $\Gamma/m \sim 0.06$, while CMS data prefer a narrow width.
Naturally, further data collected at the LHC should provide a clearer picture as
to the nature of this excess.

There has been vast interest in the possibility that the diphoton excess results
from physics beyond the SM. Most discussion has focused on models where the
excess is due to a new scalar particle which subsequently decays into two
photons \textit{e.g.}\ Ref.~\cite{Franceschini:2015kwy}. The possibility that
the new scalar particle is a bound state of exotic charged fermions has also
been considered, \textit{e.g.}\ Refs.~\cite{Kats:2016kuz, Curtin:2015jcv,
  Kamenik:2016izk, Ko:2016sht, Barrie:2016ndh}. Here we consider the case that
the \SI{750}{\GeV} state is a non-relativistic bound state constituted by an
exotic \textit{scalar} particle $\chi$ and its antiparticle, charged under
$\mathrm{SU}(3)_{c}$ as well as a new unbroken non-abelian gauge interaction.
Having $\chi$ be a scalar rather than a fermion is not merely a matter of taste:
In such a framework a fermionic $\chi$ would lead to the formation of bound
states which (typically) decay to dileptons more often than to photons; a
situation which is not favoured by the data.

The bound state, which we denote $\Pi$, can be produced through gluon--gluon
fusion directly (\textit{i.e.}\ at threshold $\sqrt{s_{gg}} \simeq M_\Pi$) or
indirectly via
$gg \rightarrow \chi^\dagger \chi \rightarrow \Pi + \textit{soft quanta}$
(\textit{i.e.}\ above $\Pi$ threshold: $\sqrt{s_{gg}} > M_\Pi$). The indirect
production mechanism can dominate the production of the bound state, which is an
interesting feature of this kind of theory.

\section{The model}

We take the new confining unbroken gauge interaction to be $\mathrm{SU}(N)$, and
assume that, like $\mathrm{SU}(3)_{c}$, it is asymptotically free and confining
at low energies. However, the new $\mathrm{SU}(N)$ dynamics is qualitatively
different from QCD as all the matter particles [assumed to be in the fundamental
representation of $\mathrm{SU}(N)$] are taken to be much heavier than the
confinement scale, $\Lambda_{N}$. In fact we here consider only one such matter
particle, $\chi$, so that $M_\chi \gg \Lambda_{n}$ is assumed. In this
circumstance a $\chi^\dagger \chi$ pair produced at the LHC above the threshold
$2M_\chi$ but below $4M_\chi$ cannot fragment into two jets. The
$\mathrm{SU}(N)$ string which connects them cannot break as there are no light
$\mathrm{SU}(N)$-charged states available. This is in contrast to heavy quark
production in QCD where light quarks can be produced out of the vacuum enabling
the color string to break. The produced $\chi^\dagger\chi$ pair can be viewed as
a highly excited bound state, which de-excites by $\mathrm{SU}(N)$-ball and soft
glueball/pion emission~\cite{Carlson:1991zn}.

With the new unbroken gauge interaction assumed to be $\mathrm{SU}(N)$ the gauge
symmetry of the SM is extended to
\begin{equation}
  \label{eq:gaugegroup}
  \mathrm{SU(3)}_{c} \otimes \mathrm{SU}(2)_{L} \otimes \mathrm{U}(1)_{Y} \otimes \mathrm{SU}(N).
\end{equation}
This kind of theory can arise naturally in models which feature large colour
groups~\cite{Foot:1990jm, Foot:2011xu, Gherghetta:2016fhp} and in models with
leptonic colour~\cite{Foot:1990dw, Foot:1991fk, Foot:2006ie, Clarke:2011aa} but
was also considered earlier by Okun~\cite{Okun:1980mu}. The notation
\textit{quirks} for heavy particles charged under an unbroken gauge symmetry
(where $M_\chi \gg \Lambda_{\textsc{n}}$) was introduced
in~\cite{Carlson:1991zn} where the relevant phenomenology was examined in some
detail in a particular model\footnote{Some other aspects of such models have
  been discussed over the years, including the possibility that the
  $\mathrm{SU}(N)$ confining scale is low ($\sim$ \keV), a situation which leads
  to macroscopic strings~\cite{Kang:2008ea}.}. For convenience we borrow their
nomenclature and call the new quantum number \textit{hue} and the massless gauge
bosons \textit{huons} ($\mathcal{H}$).

The phenomenological signatures of the bound states (quirkonia) formed depend on
whether the quirk is a fermion or boson. Here we assume that the quirk $\chi$ is
a Lorentz scalar in light of previous work which indicated that bound states
formed from a fermionic $\chi$ state would be expected to be observed at the LHC
via decays of the spin-1 bound state into opposite-sign lepton pairs
($\ell^+\ell^-$)~\cite{Carlson:1991zn, Clarke:2011aa}. In fact, this appears to
be a serious difficulty in attempts to interpret the \SI{750}{\GeV} state as a
bound state of fermionic quirk particles (such as those of
Refs.~\cite{Kats:2016kuz, Curtin:2015jcv, Kamenik:2016izk}). The detailed
consideration of a scalar $\chi$ appears to have been largely
overlooked\footnote{The idea has been briefly mentioned in recent
  literature~\cite{Agrawal:2015dbf, Ko:2016sht}.}, perhaps due to the paucity of
known elementary scalar particles. With the recent discovery of a Higgs-like
scalar at \SI{125}{\GeV}~\cite{Aad:2012tfa, Chatrchyan:2012ufa} it is perhaps
worth examining signatures of scalar quirk particles. In fact, we point out here
that the two photon decay is the most important experimental signature of bound
states formed from electrically charged scalar quirks. Furthermore this
explanation is only weakly constrained by current data and thus appears to be a
simple and plausible option for the new physics suggested by the observed
diphoton excess.
