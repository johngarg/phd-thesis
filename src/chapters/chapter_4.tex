% path to figures directory
\graphicspath{{img/chapter_4/}}


\chapter{The $S_{1}$ leptoquark as an explanation of the flavour anomalies}
\label{chapter:the-one-lq}

\begin{flushleft}
  \textit{This chapter is based on the publication `Reconsidering the One
    Leptoquark scenario: flavour anomalies and neutrino mass,' written in
    collaboration with Yi Cai, Michael A. Schmidt, and Raymond R.
    Volkas~\cite{Cai:2017wry}. We study the potential of the $S_{1}$ leptoquark
    to explain the flavour anomalies and the anomalous magnetic moment of the
    muon in a new region of parameter space.}
\end{flushleft}

\section{Introduction}

A common origin for $R_{D^{(*)}}$ and the anomalous $b\rightarrow s$ data is
suggested naturally if the former is explained by the effects of the operator
$(c^{\dagger} \bar{\sigma}_\mu b)(\tau^{\dagger} \bar{\sigma}^\mu \nu)$, related in its
general structure by $\mathrm{SU}(2)_L$ invariance to the aforementioned
four-fermion effective operator accounting for the $b \rightarrow s$ anomalies.
A number of models exploring this idea have been suggested in the
literature~\cite{Alonso:2015sja, Bauer:2015knc, Becirevic:2016oho,
  Becirevic:2016yqi, Boucenna:2016wpr, Boucenna:2016qad, Calibbi:2015kma,
  Crivellin:2017zlb, Deppisch:2016qqd, Deshpand:2016cpw, Fajfer:2015ycq,
  Feruglio:2016gvd, Feruglio:2017rjo, Megias:2017ove, Popov:2016fzr} (along
with many others addressing one or the other anomaly,
\textit{e.g.}~\cite{Becirevic:2015asa, Becirevic:2017jtw, Buras:2013qja,
  Freytsis:2015qca, Gauld:2013qba, Glashow:2014iga, Gripaios:2014tna,
  Hiller:2014ula, Hiller:2014yaa, Mahmoudi:2014mja, Megias:2016bde, Pas:2015hca,
  Sahoo:2015fla, Sahoo:2015qha, Sakaki:2013bfa, Sierra:2015fma,
  Varzielas:2015iva, deBoer:2015boa}) and among these minimal explanations the
Bauer--Neubert (BN) model~\cite{Bauer:2015knc} is one of notable simplicity and
explanatory power: a \TeV-scale scalar leptoquark protagonist mediating
$\bar{B}\rightarrow D^{(*)}\tau \bar{\nu}$ at tree-level and the
$b \rightarrow s$ decays through one-loop box diagrams. The leptoquark
transforms under the SM gauge group like a right-handed down-type quark and its
pattern of couplings to SM fermions can also reconcile the measured and
predicted values of the anomalous magnetic moment of the muon, another enduring
tension.

% Our aim in this work is twofold: (i) to study the scalar leptoquark model in the
% context of some previously unconsidered constraints and comment more definitely
% on its viability as an explanation of both $R_{D^{(*)}}$ and $R_{K^{(*)}}$; and
% (ii) to build on previous work by considering a two-loop neutrino mass model
% (first presented in Ref.~\cite{Angel:2013hla}) whose particle content includes
% the TeV-scale scalar leptoquark present in the BN scenario. In doing so we hope
% to establish the explanatory power of this simple extension of the SM,
% emphasizing the simplicity with which it can be embedded into a radiative model
% of Majorana neutrino mass. We find that the two-loop scheme heavily alleviates
% the fine-tuning present in the one-loop models, and we expect this result to be
% general for all two-loop topologies.

\index{bold}
\myglossaryentry{lipsum}{lipsum}{Lorem Ipsum, a special type of fudge}{}
\myglossaryentry{dolor}{dolor}{No idea why}{parent={lipsum}}
\myglossaryentry{ibit}{ibit}{Sounds right, doesn't it?}{parent={lipsum}}
\myacronym{DFT}{density functional theory}
\myglossaryentry{$\pi$}{pi}{Greek letter pi, \ensuremath{\Pi} does this work?}{symbol={$\pi$}}
\myacronym{RDF}{radial distribution function}
\myglossaryentry{radial distribution function}{radialdistributionfunction}{}{symbol={$g(r)$}}
