%%
%% VERSION HISTORY
%%    22 May 2006 - John Papandriopoulos - Original version
%%    12 Jul 2007 - John Papandriopoulos - Converted into template
%%    30 Sep 2015 - Maoyuan Liu - Draft version
%%    08 Aug 2020 - John Gargalionis - Draft version

% -----
% Basic
% -----

\usepackage{amsmath}
\usepackage{mathtools}
\usepackage{epsfig}
\usepackage{graphicx}
\usepackage[numbers,sort&compress]{natbib}
\usepackage{color}
\usepackage[colorlinks=true
,urlcolor=blue
,anchorcolor=blue
,citecolor=blue
,filecolor=blue
,linkcolor=blue
,menucolor=blue
,pagecolor=blue
,linktocpage=true
,pdfproducer=medialab
,pdfa=true
]{hyperref}
\usepackage{url}
\usepackage{enumitem}
\usepackage{mathtools}
\usepackage{simplewick}
\usepackage{simpler-wick}
\usepackage{pifont}
\newcommand{\cmark}{\text{\ding{51}}}
\newcommand{\xmark}{\text{\ding{55}}}

\usepackage{caption}
\usepackage{subcaption}
\usepackage{longtable}
\usepackage[version=3]{mhchem}

\usepackage{array}
\usepackage{makecell}
\newcolumntype{B}{>{\centering\arraybackslash}m{6cm}}
\newcolumntype{M}{>{\centering\arraybackslash}m{3cm}}
\newcolumntype{S}{>{\centering\arraybackslash}m{1.5cm}}

\usepackage{booktabs}
\usepackage{multirow}
\usepackage{colortbl}
\usepackage{tablefootnote}

% math
\usepackage{slashed}

% Units
\usepackage[alsoload=hep,detect-all]{siunitx}
\sisetup{mode=math}

% For sans serif figure captions
\usepackage[font=sf]{caption}

% Index
\usepackage{makeidx}
\makeindex

% Glossary --> defitions of symbols and acronyms
\usepackage[toc,nonumberlist,nopostdot,abbreviations]{glossaries-extra}
\makeglossaries
\setglossarystyle{treegroup}
\newacronym{EFT}{EFT}{Effective Field Theory}


% ------
% Layout
% ------

% Suppress "This page intentionally left blank."
\newcommand*\markblankpages{}

% set page margins
%   these margins are set by the A4 paper ratios, 1:sqrt(2)
%   paperheight/paperwidth = paperwidth/textwidth = paperheight/textheight = sqrt(2)
%   outer/inner = bottom/top = sqrt(2)

%   Original version (top and bottom values for \symmetricmargin and \bookmargin
%   all the same)
%   \usepackage[outer=36.02988mm,inner=25.47697mm,top=36.0317mm,bottom=50.9565mm]{geometry}
\usepackage[outer=36.02988mm,inner=25.47697mm,top=36.0317mm,bottom=50.9565mm]{geometry}
\newcommand*\symmetricmargin{
    \newgeometry{outer=30.7534mm,inner=30.7534mm,top=36.0317mm,bottom=50.9565mm}
}
\newcommand*\bookmargin{
  \newgeometry{outer=36.02988mm,inner=25.47697mm,top=36.0317mm,bottom=50.9565mm}
}

% For testing the formatting
\usepackage{lipsum}

\usepackage{tikz}
\usepackage{tikz-feynman}

% ---------
% Eye candy
% ---------
\usepackage[labelfont={bf,singlespacing},
            textfont={singlespacing},
            justification={justified,RaggedRight},
            singlelinecheck=false,
            margin=0pt,
            figurewithin=chapter,
            tablewithin=chapter]{caption}
\usepackage[titletoc]{appendix}

% fancy chapter titles and quotes at beginning of title
\usepackage[helvetica]{quotchap} 

% Draft watermark
\usepackage[contents={}]{background}
\usepackage[yyyymmdd,hhmmss]{datetime}
\newcommand*\DraftText{Draft compiled on \today\ at \currenttime}
\newcommand*\draft{
  \newcommand*{\archivalpapernote}{}
  \backgroundsetup{
    color=lightgray,
    position=current page.center,
    angle=90,
    vshift=0.45\paperwidth,
    opacity=1,
    scale=1,
    contents={\LARGE\ttfamily\DraftText}
  }
}

% numbers in LNV table
\newcommand{\mynum}[1]{
  \num[
  scientific-notation = true,
  round-mode = figures,
  round-precision = 1,
  exponent-product = \cdot
  ]{#1}
}
