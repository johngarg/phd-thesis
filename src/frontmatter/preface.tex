\begin{preface}
  Particle physics currently finds itself in a strange or exciting place,
  depending on who you ask. The discovery of a Higgs-like boson at close to
  $\SI{125}{\GeV}$ has meant both the completion of the Standard Model (SM), and
  the end of clear signs of new particles at the electroweak scale. Although the
  Large Hadron Collider (LHC) will continue to collect data well into the next
  few decades, the mass reach will not increase significantly. The community
  waits for a new machine, for which there are many candidates and promises,
  that will continue to push the energy frontier and test theories addressing
  the many shortcomings of the SM. Time frames for many of these see data taking
  beginning at the end of my career. If progress is driven by experiment, where
  do we go from here?

  Thankfully, there are already clear signs of new physics in the neutrino
  sector. The observation of neutrino oscillations, and therefore neutrino
  masses, is by far the strongest terrestrial evidence demanding an extension of
  the SM. It is no surprise that a full understanding of the neutrinos has
  alluded us so far; they are, with the possible exception of the Higgs boson,
  the most elusive particles currently under laboratory scrutiny. As we move
  into an era of precision neutrino measurements, now is the right time to take
  stock of the phenomenologically viable and economic models that explain the
  pattern of neutrino masses and mixings observed. Armed with the list of
  possible mechanisms, we can make progress in probing those that are testable
  and, given that these models are falsified, build circumstantial evidence in
  favour of those that are not.

  Even on the collider front, it is unclear yet that the LHC has left us with
  the so-called `nightmare scenario' of a lonely Higgs. Perhaps unexpectedly,
  the most interesting signs of new physics from CERN have come from the
  LHC\textit{b} experiment. The now famous `flavour anomalies' are a collection
  of theoretically consistent anomalous measurements indicating a departure from
  the lepton-flavour universality present in the SM. Are these related to the
  growing evidence for deviations in leptonic anomalous magnetic moments? Might
  they be clues to a deeper theory of flavour and mass? The Belle II experiment
  has only just begun taking data, and we wait eagerly for what it has to say on
  these matters. LHC\textit{b} too will continue to improve its measurements
  with more collisions; if the anomalies persist, these will be undeniable
  evidence of physics beyond the SM accessible to the next generation of hadron
  colliders.

  These measurements are tantalising because of their consistency and breadth,
  but it would not be the first time that physicists have been lead astray,
  should they disappear with more statistics. Even so, what is perhaps the
  central result of my doctoral work will remain unchanged: that deviations from
  lepton-flavour universality in four-fermion operators may be intimately
  connected to mass generation in the neutrino sector.

\end{preface}
