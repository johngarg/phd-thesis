\begin{abstract}%
  This thesis presents a series of original studies exploring the space of
  neutrino-mass models, and the connection that a class of these models might
  have with the recently purported violations of lepton flavour universality
  measured in $B$-meson decays.

  We begin by describing and implementing an algorithm that systematises the
  process of building models of Majorana neutrino mass starting from effective
  operators that violate lepton number by two units. We use the algorithm to
  generate computational representations of all of the tree-level completions of
  the operators up to and including mass-dimension eleven, almost all of which
  correspond to models of radiative neutrino mass. Our study includes
  lepton-number-violating operators involving derivatives, updated estimates for
  the bounds on the new-physics scale associated with each operator, an analysis
  of various features of the models, and a look at some examples. Accompanying
  this work we also make available a searchable database containing the
  catalogue of neutrino-mass models, as well as the code used to find the
  completions.

  The anomalies in $B$-meson decays have known explanations through exotic
  scalar leptoquark fields. We add to this work by presenting a detailed
  phenomenological analysis a particular scalar leptoquark model: that
  containing $S_{1} \sim (\mathbf{3}, \mathbf{1}, -\tfrac{1}{3})$. We find that
  the leptoquark can accommodate the persistent tension in the ratios
  $R_{D^{(*)}}$ as long as its mass is lower than approximately $\SI{10}{\TeV}$,
  and show that a sizeable Yukawa coupling to the right-chiral tau lepton is
  necessary for an acceptable explanation. Agreement with the measured
  $R_{D^{(*)}}$ values is mildly compromised for parameter choices addressing
  the tensions in the $b \to s$ transition. The leptoquark can also reconcile
  the predicted and measured value of the anomalous magnetic moment of the muon,
  and appears naturally in models of radiative neutrino mass. As a
  representative example, we incorporate the field into a two-loop neutrino mass
  model from our database. In this specific case, the structure of the
  neutrino-mass matrix provides enough freedom to explain the small masses of
  the neutrinos in the region of parameter space dictated by agreement with the
  anomalies in $R_{D^{(*)}}$, but not in the $b \to s$ transition.

  In order to address the shortcomings of the $S_{1}$ scenario, we construct a
  non-minimal model containing the scalar leptoquarks $S_{1}$ and
  $S_{3} \sim (\mathbf{3}, \mathbf{3}, -\tfrac{1}{3})$ along with a vector-like
  quark, necessary for lepton-number violation. We find that this new model
  permits a simultaneous explanation of all of the flavour anomalies in a region
  of parameter space that also reproduces the measured pattern of neutrino
  masses and mixing. A characteristic prediction of our model is a rate of
  muon--electron conversion in nuclei fixed by the $b \to s$ anomalies and the
  neutrino mass. The next generation of muon--electron conversion experiments
  will thus potentially discover or falsify our scenario.

  Finally, we present a general overview from our model database of those
  minimal radiative neutrino-mass models that contain leptoquarks that are known
  to explain the anomalies in $R_{D^{(*)}}$ and the $b \to s$ transition. We
  hope that our model database can facilitate systematic analyses similar to
  this, perhaps on both the phenomenological and experimental fronts.
\end{abstract}
