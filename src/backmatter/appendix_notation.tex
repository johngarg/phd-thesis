%% path to figures directory
% \graphicspath{{img/chapter_4/}}

\chapter{Mathematical notation}
\label{chapter:notation}

Throughout this thesis we choose to label representations by their dimension,
which we typeset in bold. Fields are labelled by their transformation properties
under the Lorentz group and the SM gauge group
$\mathrm{SU}(3)_{c} \otimes \mathrm{SU}(2)_{L} \otimes \mathrm{U}(1)_{Y}$. All
spinors are treated as two-component objects transforming as either
$(\mathbf{2}, \mathbf{1})$ (left-handed) or $(\mathbf{1}, \mathbf{2})$
(right-handed) under the Lorentz group, written as
$\mathrm{SU}(2)_{+} \otimes \mathrm{SU}(2)_{-}$. The left-handed spinors carry
undotted spinor indices $\alpha, \beta, \ldots \in \{1, 2\}$, while the
right-handed spinors carry dotted indices
$\dot{\alpha}, \dot{\beta}, \ldots \in \{\dot{1}, \dot{2}\}$. Wherever possible
we attempt to conform to the conventions of Ref.~\cite{Dreiner:2008tw} when
working with spinor fields (see appendix G for the correspondence to
four-component notation and appendix J for SM-fermion nomenclature). For objects
carrying a single spacetime index $V_\mu$ we define
\begin{equation}
  V_{\alpha \dot{\beta}} = \sigma^\mu_{\alpha \dot{\beta}} V_\mu \quad \text{
    and
  } \quad \bar{V}_{\dot{\alpha}\beta } = \bar{\sigma}^\mu_{\dot{\alpha}\beta} V_{\mu} \ .
\end{equation}
Note that in this notation
\begin{equation}
  \Box = \partial_{\mu} \partial^{\mu} = \tfrac{1}{2}\text{Tr}[\partial \bar{\partial}] = \tfrac{1}{2}\text{Tr}[\bar{\partial} \partial] \ ,
\end{equation}
and we will often just use $\Box$ to represent the contraction of two covariant
derivatives $D_{\mu}D^{\mu}$ where this is clear from context. For
field-strength tensors, generically $X_{\mu\nu}$, we work with the irreducible
representations (irreps) $X_{\alpha \beta}$ and
$\bar{X}_{\dot{\alpha} \dot{\beta}}$, where
\begin{equation}
  X_{\{\alpha \beta\}} = 2i [\sigma^{\mu \nu}]^{~\gamma}_\alpha \epsilon_{\gamma \beta} X_{\mu \nu} \quad \text{ and } \quad
  \bar{X}_{\{\dot{\alpha} \dot{\beta}\}} = 2i [\bar{\sigma}^{\mu \nu}]^{\dot{\gamma}}_{~\dot{\beta}} \epsilon_{\dot{\alpha} \dot{\gamma}} X_{\mu \nu} \ ,
\end{equation}
or the alternate forms with one raised and one lowered index. We also define
\begin{equation}
  \label{eq:app-bidirectional-deriv}
  \phi_{1} \tilde{D}^{\mu} \phi_{2} = \frac{1}{2} [\phi_{1} D^{\mu} \phi_{2} - (D^{\mu}\phi_{1})\phi_{2}] \ ,
\end{equation}
for some fields $\phi_{i}$. Where we use four-component spinor fields, we always
simplify $\overline{\chi^{C}}\psi$ (where $\chi^{C}$ is the charge-conjugate
spinor) to $\chi \psi$ to avoid clutter.

Indices for $\mathrm{SU}(2)_{L}$ (isospin) are taken from the middle of the
Latin alphabet. These are kept lowercase for the fundamental representation for
which $i, j, k, \ldots \in \{1, 2\}$ and the indices of the adjoint are
capitalised $I, J, K, \ldots \in \{1, 2, 3\}$. Colour indices are taken from the
beginning of the Latin alphabet and the same distinction between lowercase and
uppercase letters is made. For both $\mathrm{SU}(2)$ and $\mathrm{SU}(3)$, a
distinction between raised and lowered indices is maintained such that, for
example, $(\psi^i)^\dagger = (\psi^\dagger)_i$ for an isodoublet field $\psi$.
However, we often specialise to the case of only raised, symmetrised indices for
$\mathrm{SU}(2)$, and use a tilde to denote a conjugate field whose
$\mathrm{SU}(2)_{L}$ indices have been raised:
\begin{equation}
  \label{eq:su2l-conj}
  \tilde{\psi}^{i} \equiv  \epsilon^{i j}\psi^{\dagger}_{j}.
\end{equation}
We adopt this notation from the usual definition of $\tilde{H}$, and note that
throughout the paper we freely interchange between $\tilde{\psi}^{i}$ and
$\psi^{\dagger}_{i}$. For the sake of tidiness, we sometimes use parentheses
$(\cdots)$ to indicate the contraction of suppressed indices. Curly braces are
reserved to indicate symmetrised indices $\{\cdots\}$ and square brackets
enclose antisymmetrised indices $[\cdots]$, but this notation is avoided when
the permutation symmetry between indices is clear. We use $\tau^I$ and
$\lambda^A$ for the Pauli and Gell-Mann matrices, and normalise the non-abelian
vector potentials of the SM such that
\begin{equation}
  (W_{\alpha \dot{\beta}})^i_{\ j} = \frac{1}{2} (\tau^I)^i_{\ j} W^I_{\alpha
    \dot{\beta}} \quad \text{ and } \quad (G_{\alpha \dot{\beta}})^a_{\ b} =
  \frac{1}{2} (\lambda^A)^a_{\ b} G^A_{\alpha \dot{\beta}}.
\end{equation}
Flavour (or family) indices of the SM fermions are represented by the lowercase
Latin letters $\{r, s, t, u, v, w\}$.

For the non-gauge degrees of freedom in the SM we capitalise isospin doublets
($Q$, $L$, $H$), while the left-handed isosinglets are written in lowercase with
a bar featuring as a part of the name of the field ($\bar{u}$, $\bar{d}$,
$\bar{e}$). The representations and hypercharges for the SM field content are
summarised in Table~\ref{tbl:sm}. Our definition of the SM gauge-covariant
derivative is exemplified by
\begin{equation}
  \label{eq:covdi}
  \bar{D}_{\dot{\alpha}\beta} Q^{\beta a i}_r = \left[ \delta^a_b \delta^i_j (\bar{\partial}_{\dot{\alpha}\beta} + i g_1 Y_Q \bar{B}_{\dot{\alpha} \beta}) + i g_2 \delta^a_b (\bar{W}_{\dot{\alpha} \beta})^i_{\ j} + i g_3 \delta^i_j (\bar{G}_{\dot{\alpha}\beta})^a_{\ b} \right] Q_r^{\beta b j} \ .
\end{equation}
Note that the derivative implicitly carries $\mathrm{SU}(2)_{L}$ and
$\mathrm{SU}(3)_{c}$ indices [explicit on the right-hand side of
Eq.~\eqref{eq:covdi}] which are suppressed on the left-hand side to reduce
clutter. Where appropriate we show these indices explicitly.

We represent the SM quantum numbers of fields as a 3-tuple
$(\mathbf{C}, \mathbf{I}, Y)_{L}$, with $\mathbf{C}$ and $\mathbf{I}$ the
dimension of the colour and isospin representations, $Y$ the hypercharge of the
field, and $L$ an (often omitted) label of the Lorentz representation: $S$
(scalar), $F$ (fermion) or $V$ (vector), although sometimes we use the irrep,
\textit{e.g.} $(\mathbf{2}, \mathbf{1})$. We normalise the hypercharge such that
$Q = I_{3} + Y$. Finally, for exotic fields that contribute to dimension-six
operators at tree-level, we try and adopt names consistent with Table 3 of
Ref.~\cite{deBlas:2017xtg}, which we reproduce here in
Table~\ref{tab:appA-field-labels}.

\begin{table}[t]
  \centering
  \bgroup
  \def\arraystretch{1.5}%  1 is the default, change whatever you need
  \begin{tabular}{ccc}
    \toprule
    Field                        & $\mathrm{SU}(3)_{c} \otimes \mathrm{SU}(2)_{L} \otimes \mathrm{U}(1)_{Y}$ & $\mathrm{SU}(2)_{+} \otimes \mathrm{SU}(2)_{-}$ \\
    \midrule
    $Q^{\alpha a i}$             & $(\mathbf{3}, \mathbf{2}, \tfrac{1}{6})$                                  & $(\mathbf{2}, \mathbf{1})$                      \\
    $L^{\alpha i}$               & $(\mathbf{1}, \mathbf{2}, -\tfrac{1}{2})$                                 & $(\mathbf{2}, \mathbf{1})$                      \\
    $\bar{u}^{\alpha}_a$                  & $(\bar{\mathbf{3}}, \mathbf{1}, -\tfrac{2}{3})$                           & $(\mathbf{2}, \mathbf{1})$                      \\
    $\bar{d}^{\alpha}_a$                  & $(\bar{\mathbf{3}}, \mathbf{1}, \tfrac{1}{3})$                            & $(\mathbf{2}, \mathbf{1})$                      \\
    $\bar{e}^{\alpha}$                    & $(\mathbf{1}, \mathbf{1}, 1)$                                             & $(\mathbf{2}, \mathbf{1})$                      \\
    $(G_{\alpha \beta})^a_{\ b}$ & $(\mathbf{8}, \mathbf{1}, 0)$                                             & $(\mathbf{3}, \mathbf{1})$                      \\
    $(W_{\alpha \beta})^i_{\ j}$ & $(\mathbf{1}, \mathbf{3}, 0)$                                             & $(\mathbf{3}, \mathbf{1})$                      \\
    $B_{\alpha \beta}$           & $(\mathbf{1}, \mathbf{1}, 0)$                                             & $(\mathbf{3}, \mathbf{1})$                      \\
    $H^{i}$                      & $(\mathbf{1}, \mathbf{2}, \tfrac{1}{2})$                                  & $(\mathbf{1}, \mathbf{1})$                      \\
    \bottomrule
  \end{tabular}
  \egroup
  \caption[The SM fields and their transformation properties under the SM gauge
  group $G_{\text{SM}}$ and the Lorentz group.]{The SM fields and their
    transformation properties under the SM gauge group $G_{\text{SM}}$ and the
    Lorentz group. The final unbolded number in the 3-tuples of the
    $G_{\text{SM}}$ column represents the $\mathrm{U}(1)_Y$ charge of the field,
    normalised such that $Q = I_{3} + Y$. For the fermions a generational index
    has been suppressed.}
  \label{tbl:sm}
\end{table}

\begin{table}[t]
  \begin{center}
    {\small
      \begin{tabular}{lcccccccc}
        \toprule
        Name &
        ${\cal S}$ &
        ${\cal S}_1$ &
        ${\cal S}_2$ &
        $\varphi$ &
        $\Xi$ &
        $\Xi_1$ &
        $\Theta_1$ &
        $\Theta_3$ \\
        Irrep &
        $(\mathbf{1},\mathbf{1},0)$ &
        $(\mathbf{1},\mathbf{1},1)$ &
        $(\mathbf{1},\mathbf{1},2)$ &
        $(\mathbf{1},\mathbf{2},{\tfrac 12})$ &
        $(\mathbf{1},\mathbf{3},0)$ &
        $(\mathbf{1},\mathbf{3},1)$ &
        $(\mathbf{1},\mathbf{4},{\tfrac 12})$ &
        $(\mathbf{1},\mathbf{4},{\tfrac 32})$ \\[1.3mm]
        %
        \midrule
        %
        Name &
        ${\omega}_{1}$ &
        ${\omega}_{2}$ &
        ${\omega}_{4}$ &
        $\Pi_1$ &
        $\Pi_7$ &
        $\zeta$ &
        & \\
        Irrep &
        $(\mathbf{\bar{3}},\mathbf{1},{\tfrac 13})$ &
        $(\mathbf{3},\mathbf{1},{\tfrac 23})$ &
        $(\mathbf{\bar{3}},\mathbf{1},{\tfrac 43})$ &
        $(\mathbf{3},\mathbf{2},{\tfrac 16})$ &
        $(\mathbf{3},\mathbf{2},{\tfrac 76})$ &
        $(\mathbf{\bar{3}},\mathbf{3},{\tfrac 13})$ \\[1.3mm]
        %
        \midrule
        %
        Name &
        $\Omega_{1}$ &
        $\Omega_{2}$ &
        $\Omega_{4}$ &
        $\Upsilon$ &
        $\Phi$ &
        &
        & \\
        Irrep &
        $(\mathbf{6},\mathbf{1},{\tfrac 13})$ &
        $(\mathbf{\bar{6}},\mathbf{1},{\tfrac 23})$ &
        $(\mathbf{6},\mathbf{1},{\tfrac 43})$ &
        $(\mathbf{6},\mathbf{3},{\tfrac 13})$ &
        $(\mathbf{8},\mathbf{2},{\tfrac 12})$ \\[1.3mm]
        %
        \bottomrule
        \toprule
        Name &
        $N$ & $E$ & $\Delta_1$ & $\Delta_3$ & $\Sigma$ & $\Sigma_1$ & \\
        Irrep &
        $(\mathbf{1}, \mathbf{1},0)$ &
        $(\mathbf{1}, \mathbf{1},{1})$ &
        $(\mathbf{1}, \mathbf{2},{\frac{1}{2}})$ &
        $(\mathbf{1}, \mathbf{2},{\frac{3}{2}})$ &
        $(\mathbf{1}, \mathbf{3},0)$ &
        $(\mathbf{1}, \mathbf{3},{1})$ & \\[1.3mm]
        %
        \midrule
        %
        Name &
        $U$ & $D$ & $Q_1$ & $Q_5$ & $Q_7$ & $T_1$ & $T_2$ \\
        Irrep &
        $(\mathbf{3}, \mathbf{1},{\frac{2}{3}}),$ &
        $(\mathbf{\bar{3}}, \mathbf{1},{\frac{1}{3}})$ &
        $(\mathbf{3}, \mathbf{2},{\frac{1}{6}})$ &
        $(\mathbf{3}, \mathbf{2},{-\frac{5}{6}})$ &
        $(\mathbf{3}, \mathbf{2},{\frac{7}{6}})$ &
        $(\mathbf{\bar{3}}, \mathbf{3},{\frac{1}{3}})$ &
        $(\mathbf{3}, \mathbf{3},{\frac{2}{3}})$ \\ [1.3mm]
        \bottomrule
      \end{tabular}
    }
    \end{center}
    \caption[The table shows the exotic scalars (top) and vectorlike or Majorana
    fermions (bottom) contributing to the dimension-six SMEFT at
    tree-level~\cite{deBlas:2017xtg}.]{The table shows the exotic scalars (top)
      and vectorlike or Majorana fermions (bottom) contributing to the
      dimension-six SMEFT at tree-level~\cite{deBlas:2017xtg}. We sometimes use
      the label of a field as presented in the table to represent its conjugate,
      although we always define the transformation properties each time a field
      is mentioned to avoid confusion. For the leptoquarks (second row), we add
      a prime to the field name presented here if the baryon-number assignment
      is such that only the diquark couplings are allowed.}
    \label{tab:appA-field-labels}
  \end{table}
